\chapter{Procedimiento}
\label{ch:procedimiento}

%! Mínimo 9 páginas. Añadir algo de lo de SCRUM de la plantilla.
%! A 29-05 van 5, faltan 4

%!
\section{SCRUM}
Sección SCRUM

%! Revisar
\section{Diseño de módulos para la \emph{FPGA}}

%* Comprobar
Durante el proceso de diseño del sistema de la \emph{FPGA}, y ante la necesidad de crear módulos o bloques funcionales de utilidad, se siguen una serie de procedimientos, definidos al comienzo de esta fase.

Gracias a su uso, a parte de establecer unos mínimos requerimientos de calidad, se obtienen con mayor eficiencia los resultados buscados, permitiendo a su vez una alta legibilidad del código, y una buena capacidad de modificación o reutilización.

Para garantizarlo, estos procedimientos deben de ser comunes en todos los módulos, e invariables a lo largo del proyecto.

%!
\subsection{Estructura de carpetas}

Todos los módulos están situados dentro de la carpeta \emph{./modules/}. Consta en primer lugar de un archivo de cabecera donde 
% \begin{figure}
%     \centering
%     \begin{minipage}{7cm}
%         \dirtree{%
%         .1 ./USB3300\_sniffer/.
%             .2 build/.
%             .2 build\_ok/.
%             .2 modules/.
%                 .3 nombre\_modulo/.
%                     .4 rtl/.
%                         .5 nombre\_modulo.v.
%                     .4 tb/.
%                         .5 nombre\_modulo\_tb.v.
%                     .4 Makefile.
%                     .4 nombre\_modulo.gtkw.
%                 .3 nombre\_modulo.vh.
%             .2 modules\_simulation/.
%                 .3 nombre\_modulo\_simulacion/.
%                     .4 rtl/.
%                         .5 nombre\_modulo\_simulacion.v.
%                     .4 tb/.
%                         .5 nombre\_modulo\_simulacion\_tb.v.
%                     .4 Makefile.
%                     .4 nombre\_modulo\_simulacion.gtkw.
%                 .3 nombre\_modulo\_simulacion.vh.
%             .2 rtl/.
%                 .3 bauds.vh.
%                 .3 top.pcf.
%                 .3 top.v.
%             .2 rtl\_test/.
%             .2 tb/.
%             .2 tools/.
%                 .3 gen\_bauds.py.
%                 .3 get\_divider.py.
%                 .3 serial\_control.py.
%         }
%     \end{minipage}
%     \caption{Estructura de carpetas del código en Verilog}
% \end{figure}


%! Revisar
\subsection{Etapas de diseño de un módulo}

%* Comprobar
En primer lugar, tras requerir la utilización de un módulo que logre una función específica, se analiza la posibilidad de reutilizar uno o varios módulos anteriormente realizados, evitando así un derroche innecesario de tiempo y recursos.

%* Comprobar
Seguidamente, se estudia la complejidad del módulo a desarrollar, dividiendolo, si fuera necesario, en varios submódulos con funciones más concretas que disminuyan el grado de dificultad general en su elaboración, permitiendo a su vez una mayor reutilización futura de código.

%* Comprobar
Una vez estén definidas sus partes, se realiza un pequeño esquema sobre el que partir, que de forma gráfica muestre sus diversas partes y relaciones, junto a este, si fuera necesario, se dibuja otro que muestre las etapas de la máquina de estados, pudiendo a continuación empezar con la programación del mismo.

Cada módulo dispone de un estilo común, distribuido de la siguiente forma:

\begin{enumerate}
    %* Comprobar
    \item{\textbf{Comentario de cabecera.}} Tal como se muestra en el listado \ref{src:procedimientos-verilog-cabecera}, se nombra al módulo, incluyendo una descripción breve de sus funciones. Además, se enumeran y explican todas las entradas, salidas, parámetros y estados de los que está formado.
    \begin{lstlisting}[language=Verilog,
        caption={Ejemplo de comentario de cabecera del módulo.},
        label=src:procedimientos-verilog-cabecera]
/*
 * <Nombre del modulo> module
 * <Descripcion del modulo>
 *
 * Parameters:
 *  - <Nombre del parametro>. <Descripcion del parametro>
 *  - etc ..
 *
 * Inputs:
 *  - <Nombre de la entrada>. <Descripcion de la entrada>
 *  - etc ..
 *
 * Outputs:
 *  - <Nombre de la salida>. <Descripcion de la salida>
 *  - etc ..
 *
 * States:
 *  - <Nombre del estado>. <Descripcion del estado>
 *  - etc ..
 */
    \end{lstlisting}

    %* Comprobar
    \item{\textbf{Control de reinicio síncrono/asíncrono.}} Se trata de un pequeño bloque que controla el modo de los reinicios. Si en el archivo principal se define la constante \emph{ASYNC\_RESET}, entonces todos los reinicios serán asíncronos, en caso contrario, serán síncronos a la señal de reloj (\emph{clk}).
    \begin{lstlisting}[language=Verilog,
        caption={Ejemplo de control de reinicio síncrono/asíncrono.},
        label=src:procedimientos-verilog-sync]
// X es el nombre del modulo en cada caso
`ifdef ASYNC_RESET
    `define X_ASYNC_RESET or negedge rst
`else
    `define X_ASYNC_RESET
`endif
    \end{lstlisting}
        
    %* Comprobar
    \item{\textbf{Incorporación de módulos necearios.}} Suele ser habitual, que el módulo que se está desarrollando haga uso de otros, los cuales necesitan ser incorporarlos previamente. Un ejemplo de ello se puede apreciar en el listado \ref{src:procedimientos-verilog-include-modules}.
    \begin{lstlisting}[language=Verilog,
        caption={Ejemplo de incorporación de módulos.},
        label=src:procedimientos-verilog-include-modules]
`include "./modules/nombre_del_modulo.vh"
\\ etc ..
    \end{lstlisting}
        
    %* Comprobar
    \item{\textbf{Creación del módulo.}} Se define el módulo, agrupando las entradas y salidas según características similares, posicionando en primer lugar las entradas, seguido de las salidas.
    \begin{lstlisting}[language=Verilog,
        caption={Ejemplo de creación de módulo.},
        label=src:procedimientos-verilog-create]
module nombre_modulo
    #(
      parameter nombre_parametro = valor_parametro
      \\ etc ..
     )
     (
      // <Nombre del primer grupo de entradas/salidas>
      input  wire nombre_entrada_1, // <breve descripcion>
        // etc ..
      output wire nombre_salida_1,  // <breve descripcion>
        // etc ..
     );
    \end{lstlisting}
    
    %* Comprobar
    \item{\textbf{Inicialización de módulos necesarios.}} Todos los módulos incorporados previamente se inicializan, relacionando sus entradas/salidas con el resto de señales del módulo.
    \begin{lstlisting}[language=Verilog,
        caption={Ejemplo inicialización de módulos.},
        label=src:procedimientos-verilog-mod-init]
// Ejemplo inicializando el modulo clk\_baud\_pulse, que tiene dos parametros, una entrada y dos salidas
clk_baud_pulse #(
                 .COUNTER_VAL(BAUDS),
                 .PULSE_DELAY(BAUDS/2)
                )
   clk_baud_Rx  (
                 .clk_in(clk),       // Input
                 .clk_pulse(clk_Rx), // Output
                 .enable(enable)     // Output
                );
    \end{lstlisting}
    
    %* Comprobar
    \item{\textbf{Creación de los registros de control.}} Se crean todos los registros encargados de controlar el módulo o almacenar temporalmente las variables.
    \begin{lstlisting}[language=Verilog,
        caption={Ejemplo de creación de registros.},
        label=src:procedimientos-verilog-regs]
// Control registers
reg [1:0]X_state_r   = 2'b0; // Register that stores the current X state
reg [3:0]X_counter_r = 4'b0; // Register that counts ...
reg [7:0]DATA_buff   = 8'b0; // Buffer where ...
    \end{lstlisting}

    %* Comprobar
    \item{\textbf{Creación de \textit{flags} o señales de control.}} Para poder identificar con facilidad si el sistema se encuentra en un estado concreto, se crean estas señales, cuyo valor será \textit{1} cuando la máquina de estados esté en dicho estado, y \textit{0} en caso contrario.
    \begin{lstlisting}[language=Verilog,
        caption={Ejemplo de creación de \textit{flags}.},
        label=src:procedimientos-verilog-flags]
// Flags
wire X_s_IDLE; // HIGH if X\_state\_r == X\_IDLE, else LOW
wire X_s_RUN;  // HIGH if X\_state\_r == X\_RUN,  else LOW
    \end{lstlisting}

    %* Comprobar
    \item{\textbf{Asignación de valores de salida y señales internas.}} Se asignan los valores que tomarán los diversos \emph{"wires"}, tanto de uso interno del módulo como de salida, estos pueden relacionarse con registros, salidas de módulos ya inicializados u otros \emph{"wires"}.
    \begin{lstlisting}[language=Verilog,
        caption={Ejemplo de asignación de valores.},
        label=src:procedimientos-verilog-asignacion]
// Assigns
assign X_s_IDLE = (X_state_r == X_IDLE) ? 1'b1 : 1'b0; // \#FLAG
assign X_s_RUN  = (X_state_r == X_RUN)  ? 1'b1 : 1'b0; // \#FLAG

assign nombre_salida_1 = DATA_buff[2]; // \#OUTPUT
    \end{lstlisting}
    

    %* Comprobar
    \item{\textbf{Enumeración de estados.}} Se crean tantos parámetros locales como estados tenga el módulo, cada uno con un valor identificativo único.
    \begin{lstlisting}[language=Verilog,
        caption={Ejemplo de enumeración de estados.},
        label=src:procedimientos-verilog-estados]
/// X States (See module description at the beginning of this file to get more info)
localparam X_IDLE = 1'b0;
localparam X_RUN  = 1'b1;
    \end{lstlisting}
    

    %* Comprobar
    \item{\textbf{Máquina de estados.}} Se fijan los distintos caminos que puede tomar el sistema, dependiendo de las entradas y del propio estado actual. Un pequeño ejemplo de una máquina de estados se muestra en el listado \ref{src:procedimientos-verilog-fst}
    \begin{lstlisting}[language=Verilog,
        caption={Ejemplo de máquina de estados.},
        label=src:procedimientos-verilog-fst]
always @(posedge clk `X_ASYNC_RESET) begin
    if(!rst) X_state_r <= X_IDLE;
    else begin
        case(X_state_r)
            X_IDLE: begin
                if(!nombre_entrada_1) X_state_r <= X_RUN;
                else                  X_state_r <= X_IDLE;
            end
            X_RUN: begin
                X_state_r <= X_IDLE;
            end
            default: X_state_r <= X_IDLE;
        endcase
    end
end
    \end{lstlisting}

    %! Revisar
    \item{\textbf{Actualización de registros.}} Por último, se actualizan los valores de los registros del módulo, teniendo en cuenta que se utiliza una máquina de estados \emph{Mealy} \cite{barkalov2005design}  \noWord [maquina estados mealy info footnote], es decir, que los valores dependen del estado actual y de sus entradas.
    \begin{lstlisting}[language=Verilog,
        caption={Ejemplo de actualización de registros.},
        label=src:procedimientos-verilog-act-regs]
always @(posedge clk `UART_RX_ASYNC_RESET) begin
    if(!rst) begin
        X_counter_r <= 0;
    end
    else if(X_s_IDLE) begin
        X_counter_r <= X_counter_r - 1'b1;
    end
    else if(X_s_RUN) begin
        X_counter_r <= X_counter_r + 1'b1;
    end
end
    \end{lstlisting}
\end{enumerate}


%!
\subsection{Pruebas}

Para asegurar el correcto funcionamiento de cada módulo, estos deben superar varias pruebas que comprueben cada una de sus funcionalidades. En caso de superar todas ellas satisfactoriamente, el módulo estará listo para ser usado en la síntesis de la \emph{FPGA}

Cada 

\cite{stuartsutherland2001}

Estas simulaciones se realizan a través del compilador \emph{Verilog} \noWord [ref/foot to main Verilog page or commands], encargado de transformar el codigo fuente del módulo a un ejecutable 


%!
\section{App}
Sección App



% \chapter{Procedimiento}
% \label{ch:procedimiento_plantilla}

% \info{Esta descripción está hecha a título orientativo. Puedes mejorarla con capturas de tu propio tableto Trello o con cualquier aclaración que consideres necesaria.}

% En el desarrollo de este TFG se ha utilizado una metodología ágil basada en \emph{Scrum}~\cite{scrumguide}, definida por el director.  El trabajo se ha dividido en iteraciones de dos semanas denominadas \emph{sprints}.  Las unidades de trabajo se presentan en forma de historias de usuario (\emph{user stories}) que definen mini-proyectos de muy corta duración que aportan valor al proyecto.  Es decir, cada historia de usuario cumple o ayuda a cumplir alguno de los objetivos.  Medir el valor percibido corresponde al propietario del producto (\emph{Product Owner}), que participa activamente en la planificación del proceso priorizando las unidades de trabajo.

% La utilización de una metodología ágil permite equilibrar la cantidad de trabajo y los objetivos alcanzados.  Los 12 créditos ECTS del TFG se reparten según el criterio del director para que los resultados aporten el máximo valor posible, incluso en presencia de imprevistos.

% \section{Diferencias con Scrum}

% \emph{Scrum} es una metodología estrictamente centrada en el cliente.  El cliente es el responsable de priorizar y, en cierto modo, planificar las iteraciones.  Esto garantiza que la ejecución del proyecto responde al máximo con las expectativas del cliente, aún cuando los imprevistos impidan alcanzar alguno de los objetivos iniciales.  Esta característica de \emph{Scrum} es la única que se ha intentado mantener inalterada.  Sin embargo, el TFG es un proyecto individual, lo que ha requerido modificar significativamente otros aspectos de la metodología.

% \subsection{Roles}

% La única remuneración que se obtiene con la ejecución de un TFG es la calificación de los distintos aspectos (anteproyecto, valoración del director, valoración del tribunal, etc.).  Por tanto, el cliente del TFG se compone por el director y el tribunal de la defensa.  Desgraciadamente no es posible conocer a priori el tribunal.  Por este motivo el director es el único representante del cliente en el proceso de desarrollo (\emph{Product Owner}).

% El TFG debe ser realizado de manera individual.  Por tanto, el equipo de trabajo (\emph{Team Member}) se compone exclusivamente por el autor.

% La labor de dirección del TFG se asimila a la de dirección del proyecto y, por tanto, el director también actúa como coordinador del proceso, o \emph{Scrum Master}.  Nótese que hay dos roles representados por la misma persona.  Desde un punto de vista purista esto implica que puede haber conflicto de intereses y los intereses del cliente pueden estar insuficientemente representados.  Es una limitación extrínseca, que no es posible solucionar con el proceso actual.  Aún así, el uso de una metodología ágil centrada en el cliente debe mejorar el alineamiento de intereses cuando sobrevienen problemas que afectan o pueden afectar a la consecución de alguno de los objetivos iniciales.

% \subsection{Historias de usuario}

% \emph{Scrum}, como la mayoría de los métodos ágiles, está enfocada al desarrollo de proyectos en entornos de alta incertidumbre por equipos multidisciplinares bien formados.  El desarrollo de un TFG, al tratarse de un primer proyecto profesional, también está sometido a gran cantidad de incertidumbre.  Sin embargo, no siempre se cuenta con la formación previa necesaria para abordar todos los problemas.  Esto implica que, en ocasiones, se necesita aprender o leer, sin repercusión medible en el valor percibido por el \emph{Product Owner}.  En esos casos se planifican unidades de trabajo que no corresponden estrictamente a historias de usuario en el sentido de Scrum.  Se ha intentado mantener al mínimo este tipo de historias de usuario para tener el proceso lo más controlado posible.

% Puntualmente ha sido necesario planificar historias de usuario que solo pretenden explorar opciones.  Este tipo de historias de usuario están contempladas en \emph{Scrum}, se denominan \emph{spikes}.  Sin embargo, en la ejecución de este TFG se ha procurado reducir al mínimo para que la exploración de alternativas no domine en el tiempo dedicado al TFG.

% \subsection{Planificación de sprints}

% Para la planificación y el seguimiento se ha utilizado un tablero \href{http://trello.com}{Trello}.  Los tableros Trello permiten agrupar tarjetas en una serie de listas con nombre.  Se ha utilizado el esquema propuesto en~\cite{andrewlittlefield2016}.

% El autor ha sido responsable de añadir la mayoría de las historias de usuario a la lista \emph{Backlog}.  Se trata de un proceso continuo, durante toda la ejecución del proyecto.  El director, como \emph{product owner}, prioriza las historias, moviendo las tarjetas dentro de la lista \emph{Backlog}.  Justo antes de cada iteración se realiza una reunión presencial o virtual para revisar la iteración pasada y planificar la siguiente iteración.

% Usando la técnica de \emph{planning poker} (ver~\cite{scrumguide}) se dimensionan las historias de usuario en días de trabajo.  Esta técnica consiste en un proceso de generación de consenso entre el autor y el director sobre el tiempo requerido para la ejecución de cada historia de usuario.  La unidad empleada ha sido de un día.

% El director, como \emph{product owner} traslada las tarjetas correspondientes a las primeras historias de la lista \emph{Backlog} a la lista \emph{ToDo} hasta completar los 10 días de trabajo de la iteración.

% \subsection{Flujo de trabajo}

% El flujo de trabajo diario del autor corresponde a la siguiente secuencia:

% \begin{itemize}
%     \item Dentro de la lista \emph{ToDo} puede elegirse cualquier tarjeta para trabajar en ella.  Antes de comenzar el trabajo se arrastra la tarjeta a la lista \emph{Doing}.  Esto proporciona información en tiempo real al director del progreso de la iteración.
    
%     \item Al terminar una historia de usuario la tarjeta correspondiente se arrastra a la lista \emph{QC} (quality control).
    
%     \item El director, como \emph{Scrum Master}, revisa que la historia está realmente acabada y, si así es, la traslada a la lista \emph{Done}. En caso contrario la traslada a la lista \emph{Doing} otra vez, añadiendo un comentario que lo justifica.

%     \item Si en el transcurso del trabajo se encuentra un obstáculo que impide progresar con una historia, se traslada a la lista \emph{Blocked}, añadiendo un comentario que lo justifica.
% \end{itemize}

% En todo momento es posible ver el estado global de ejecución del proyecto.  Al finalizar, la lista \emph{Done} contiene todas las historias de usuario ejecutadas por orden de terminación.  Y las listas \emph{Blocked} y \emph{Backlog} contienen (en este orden) todas las historias de usuario que corresponderían a trabajo futuro, ya priorizadas por el director.

% \subsection{Herramientas de ayuda}

% El proceso de desarrollo está fuertemente ligado a la herramienta \href{https://trello.com/}{Trello}.  Se trata de una herramienta colaborativa en línea, que permite mantener una serie de tarjetas agrupadas en listas con nombre.  Cada tarjeta puede tener un título, una descripción, un conjunto de adjuntos, y un conjunto de comentarios.  Trello se ha usado con éxito en la planificación de proyectos de nivel de complejidad muy variable.  Por ejemplo, Epic Games utiliza un tablero Trello para planificar las características a incorporar a cada nueva versión de Unreal Engine.  Por otro lado, un problema de Trello es el manejo limitado de la historia de modificaciones en las tarjetas y en los movimientos entre listas de tarjetas.  Esto dificulta en cierto modo el seguimiento de los cambios y, sobre todo, la corrección de errores en el proceso.  Por este motivo, Trello solo se ha empleado en la coordinación del trabajo, mientras que toda la gestión de cambios se ha delegado en otra herramienta.

% Todo el proyecto ha sido gestionado desde su inicio con una herramienta de control de versiones distribuido~\cite{scottchaconbenstraub2018} en un repositorio público de GitHub, disponible en \thegitrepo.  Cada vez que se completa con éxito una historia de usuario se notifica mediante un comentario en la tarjeta correspondiente.  Este comentario tan solo contiene el identificador del paquete de cambios (\emph{commit}) que da por concluida la historia.  Todos los \emph{stakeholders} pueden consultar la evolución del proyecto en todo momento desde la propia página del repositorio.

% \warning{Es posible detallar en este capítulo las iteraciones desarrolladas.  Otra posibilidad es comentar brevemente los problemas encontrados y en el siguiente capítulo explicar los resultados globales.}
