\chapter{Chuleta de \LaTeX{}}
\label{ch:latex}

\LaTeX{} es un formato textual para construir documentos complejos, cualquier tipo de documento complejo.  No es difícil, y de hecho está pensado para que personas sin formación en tipografía y composición puedan redactar documentos estéticamente impecables.  Pero necesita algo de tiempo para acostumbrarse a la filosofía.

Puedes acabar tu TFG simplemente imitando la redacción de este documento, pero seguro que puedes sacar más partido si aprendes algo de \LaTeX{}.  Hay muchas guías sencillas para empezar. Por ejemplo,~\cite{ernestoaranda2013} o~\cite{gastonsimone2002} son resúmenes suficientemente completos y sencillos.  En caso de que quieras ampliar tus conocimientos tendrás que empezar por el libro de Leslie Lamport~\cite{leslielamport1994}, el creador de \LaTeX{}.  Procura complementarlo con~\cite{latexcompanion2004} y, especialmente, navegar por el \href{https://ctan.org}{CTAN} (\emph{Comprehensive \TeX{} Archive Network}).  CTAN es un repositorio enorme de paquetes, que facilitan la edición de todo tipo de detalles.  Desde las cabeceras de página, hasta la creación de enlaces hipertexto o el manejo de colores.

En este anexo se incluye un pequeño recetario, extraído de la plantilla de Fernando Castillo, que puedes usar a modo de recordatorio rápido, pero procura utilizar una referencia más completa cuando empieces a editar el documento.

\section{Texto} 
\label{sec:texto}

\noindent En esta sección se muestran ejemplos de efectos de texto.

\begin{description}
\item[\texttt{textbf}] \textbf{Texto en negrita.}

\item[\texttt{textit}] \textit{Texto en cursiva.}

\item[\texttt{underline}] \underline{Texto subrayado.}

\item[\texttt{large}] {\large Texto grande}

\item[\texttt{Large}] {\Large Texto más grande}

\item[\texttt{LARGE}] {\LARGE Texto mucho más grande}

\item[\texttt{textsc}] \textsc{Texto en versalitas}

\item[\texttt{textsf}] \textsf{Texto en fuente sans-serif}

\item[\texttt{texttt}] \texttt{Texto en fuente mono-espaciada}
\end{description}

A veces resulta útil este tipo de texto para funciones de programación o código fuente. Por ejemplo:

\begin{verbatim}
    [x,y]=function(t,z) 
\end{verbatim}

\subsection{Listas numeradas y con viñetas} 

Listas numeradas. Se utilizan cuando la posición que ocupan los elementos es importante, para contar los elementos, o cuando se necesita añadir referencias a algunos de los elementos.

\begin{enumerate}
\item diodos
\item transistores
    \begin{enumerate}
    \item transistores pnp
    \item transistores npn
    \item operacionales
    \end{enumerate}
\item operacionales
\end{enumerate}

Listas con viñetas. Se utilizan cuando la posición concreta de los elementos no es importante:

\begin{itemize}
\item diodos
\item transistores
    \begin{itemize}
    \item transistores pnp
    \item transistores npn
    \item operacionales
    \end{itemize}
\item operacionales
\end{itemize}

\section{Figuras}
\label{sec:figuras}

En esta sección se muestran algunos ejemplos de figuras.  La fig.~\ref{fig:figurita-ejemplo} es un ejemplo de \emph{Encapsulated PostScript}.  Se trata de un formato vectorial, que no se degrada al escalarlo.  Este tipo de formatos son los ideales para usar con \LaTeX{}.  Utiliza siempre que puedas imágenes en formato EPS o PDF.

\begin{figure}[hbtp]
\centering
\includegraphics[scale=0.5]{ejemplo.eps}
\caption{Figurita ejemplo}
\label{fig:figurita-ejemplo}
\end{figure}

La forma de incluir la figura es simple:

\begin{lstlisting}[language={[LaTeX]TeX},frame=none,numbers=none]
\begin{figure}
\centering\includegraphics[width=6cm]{ejemplo.eps}
\caption{Figurita ejemplo}
\label{fig:figurita-ejemplo}
\end{figure}
\end{lstlisting}

El entorno \texttt{figure} crea un cuadro flotante, con todo el contenido de la figura, que \LaTeX{} coloca en el sitio menos malo.  La \texttt{caption} es el pie de la figura y la \texttt{label} es la etiqueta que nos permitirá referirnos a ella en el texto (con la orden \texttt{ref}).

\LaTeX{} utiliza un algoritmo nada evidente para colocar las figuras de manera que sea estéticamente agradable.  Pero tú puedes influir en las preferencias de colocación.  El entorno \texttt{figure} tiene un parámetro opcional entre corchetes que indica las opciones de colocación.  Por defecto es \texttt{[tbp]}, que equivale a \emph{top, bottom, page}.  Eso quiere decir que intenta primero ponerla a comienzo de página.  Si no lo consigue, al final de una página.  Y si así tampoco lo consigue, en una página entera, solo para la figura.  En este ejemplo utilizo las opciones \texttt{[hbtp]} para que intente la secuencia \emph{here, bottom, top, page}.  En este caso prefiero que la ponga debajo antes que arriba, para que no aparezca en una sección anterior.

Con \LaTeX{} puedes conseguir que las figuras no se muevan en absoluto, pero eso deja documentos extremadamente descompensados.  No lo hagas nunca.  Es mejor mover ligeramente la figura en el texto o incluso re-escribir parte del texto, antes de forzar la posición.  De todas formas, si no me quieres hacer caso, en la fig.~\ref{fig:figurita-ejemplo-2}  tienes un ejemplo que fuerza la posición.

\begin{figure}[h!]
\centering
\includegraphics[scale=0.5, angle=15]{ejemplo.eps}
\caption{Figurita ejemplo 2}
\label{fig:figurita-ejemplo-2}
\end{figure}

Para poder poner figuras que no son de elaboración propia es necesario primero obtener permiso del autor y, además, añadir la fuente al pie de foto. Hay muchas guías de estilo que explican en detalle cómo hacerlo. Por ejemplo, la \emph{American Psychological Association} tiene un \href{https://www.lib.sfu.ca/help/research-assistance/format-type/online-images/citing#citing-images-in-apa}{capítulo específico de su manual de publicaciones}.  El manual de la APA se usa extensivamente en todo tipo de literatura científica.

\begin{figure}[htb]
\centering
\includegraphics[scale = 0.5]{ejemplo2.jpg}
\caption{Figurita ejemplo 3. Extraída de la plantilla de TFG de Fernando Castillo. \copyright 2018 Fernando Castillo. Reproducida con permiso.}
\label{fig:figura-ejemplo-3}
\end{figure}

\warning{Fíjate bien.  No se citan las imágenes como si se tratara de referencias bibliográficas.  No debe haber pie de página (orden \texttt{footnote}) ni cita (orden \texttt{cite}) en un pie de foto (\texttt{caption}).  La atribución de la obra debe estar al mismo nivel que la obra usada.  Por eso debe atribuirse completamente en el pie de foto.  Si no te gusta como queda haz tus propias imágenes.}

Se puede controlar el escalado de la imagen y el ángulo de forma muy sencilla, con las opciones de la orden \texttt{includegraphics}.  En la fig.~\ref{fig:figura-ejemplo-3} se muestra un ejemplo de figura escalado a un 30\%.  La orden \texttt{includegraphics} ajusta los parámetros de la imagen para mantener la relación de aspecto original, si esto es posible.  Esto hace que podamos especificar simplemente el ancho o el alto deseado, que va a ser lo más habitual.  En mi opinión, las opciones más frecuentes de \texttt{includegraphics} son, por orden:

\begin{figure}
\centering
\includegraphics[width=\textwidth]{ejemplo.eps}
\caption{Figura ejemplo que ocupa todo el ancho del texto.}
\label{fig:figura-ejemplo-4}
\end{figure}

\begin{description}
\item[\texttt{width}]  Fija el ancho de la imagen.  Puede ser un tamaño absoluto en centímetros (\texttt{cm}), milímetros (\texttt{mm}) o puntos PostScript (\texttt{pt}).  Por ejemplo,  \verb|width=1.5cm|.  También puede ser un tamaño relativo a cualquier medida del documento.  Por ejemplo, \verb|width=0.5\textwidth| sería una figura que ocupe la mitad del ancho del texto.

\item[\texttt{height}] Fija la altura de la imagen.  Es similar a \texttt{width} pero con la altura.  Se puede especificar tanto altura como anchura, de manera que se modifica la relación de aspecto original.

\item[\texttt{scale}] Utiliza un factor de escala para la imagen.  Puede ser mayor de 1 para ampliar la imagen.

\item[\texttt{angle}] Gira la imagen un número de grados determinado.  Si el número es negativo el giro es en sentido horario.  Si es positivo el giro es antihorario.
\end{description}


En la fig.~\ref{fig:figura-angulo-30} se muestra un ejemplo de figura girada 30.

\begin{figure}[btp]
\centering
\includegraphics[width=.3\textwidth,angle=-30]{ejemplo.eps}
\caption{Figura ejemplo de rotación.}
\label{fig:figura-angulo-30}
\end{figure}

Una característica interesante del entorno \texttt{figure} es que permite definir sub-figuras con la orden \texttt{subfigure} o el entorno del mismo nombre.  La colocación de las sub-figuras es prácticamente automática.  Un ejemplo puede verse en la figura~\ref{fig:matriz-figuras}.

\begin{figure}[htbp]
\centering
\subfigure[figurita1]{\includegraphics[width=40mm]{ejemplo.eps}}
\subfigure[figurita2]{\includegraphics[width=40mm]{ejemplo.eps}}
\subfigure[figurita3]{\includegraphics[width=80mm]{ejemplo.eps}}
\caption{Matriz de figuras} 
\label{fig:matriz-figuras}
\end{figure}

Una subfigura puede referenciarse a partir de la referencia a la figura.  Por ejemplo, la figura~\ref{fig:matriz-figuras}(a) es igual que la figura~\ref{fig:matriz-figuras}(b). Sin embargo, cuando las figuras son muy complejas, es posible que se prefieran esquemas más automáticos.  En \href{https://tex.stackexchange.com/questions/181225/how-to-reference-to-subfigure-in-latex}{StackExchange} encontrarás soluciones a éste y otros problemas con mucha facilidad.

En general, he tratado de mantener un compromiso entre el número de características incluidas y la facilidad de uso de la plantilla.  Las figuras muy complejas es algo que prefiero evitar, así que en el estilo de la plantilla no incorpora los paquetes \texttt{subcaption} y \texttt{cleveref}.

\section{Tablas} 
\label{sec:tablas}

En esta sección se muestran algunos ejemplos de tablas.

\begin{table}[htb]
\centering
\vspace{2mm}
\begin{tabular}{|c|c|c|}
\hline
 Regulador & Función de Transferencia & orden  \\
 \hline
 P         & $\alpha_1$               & 2      \\
 \hline
\end{tabular}
\caption{Resultados de la simulación}
\label{tab:resultados-simulacion}
\end{table}

Las tablas en \LaTeX{} no son complejas pero puedes simplificar aún más usando un editor interactivo de tablas.  Por ejemplo, en \url{https://truben.no/table/} hay una aplicación \emph{online} para editar multitud de formatos de tablas.  Es especialmente útil para tablas complicadas.

En \LaTeX{} es bastante frecuente separar la cabecera del cuerpo de la tabla poniendo dos \texttt{hline}, como en la tabla~\ref{tab:sencilla}.

\begin{table}[htb]
\begin{center}
\begin{tabular}{|l|l|}
\hline
País     &    Ciudad \\ \hline \hline
España   &    Madrid \\ \hline
España   &    Valencia \\ \hline
Francia  &    París \\ \hline
\end{tabular}
\caption{Tabla muy sencilla.}
\label{tab:sencilla}
\end{center}
\end{table}

La complejidad empieza cuando hay que expandir celdas para ocupar varias columnas o varias filas.  Por ejemplo, la tabla (\ref{tab:dificililla}) tiene una celda multi-columna y otra celda multi-fila.  En estos casos un editor interactivo como el de \href{https://truben.no/table/}{Peder Lång Skeidsvoll} puede ser de gran ayuda para un principiante.  Explora las opciones, no son evidentes al principio.

\begin{table}[htb] 
\centering
\begin{tabular}{|c|c|}
\hline
\multicolumn{2}{|c|}{Europa} \\
\hline
País  & Ciudad \\ \hline \hline
\multirow{2}{1.1cm}{España} & Madrid \\ \cline{2-2}
& Valencia \\ \hline
Francia & París \\ \hline
\end{tabular}
\caption{Fusionando celdas.}
\label{tab:dificililla}
\end{table}

Las tablas, al igual que las figuras, tienen un parámetro opcional entre corchetes que indican las preferencias de posición.  Se puede forzar pero, al igual que con las figuras, conduce a documentos muy descompensados.  Procura evitarlo.  Dentro de la tabla se define un entorno \texttt{tabular} que indica con su argumento obligatorio las columnas.  Este entorno es muy útil en cualquier organización matricial.  Se puede usar también para presentar las subfiguras de una figura, o para definir una matriz.

Las tablas muy largas deben dividirse en varias páginas.  En el estilo de este TFG hemos incluído el paquete \texttt{longtable}, que facilita enormemente escribir este tipo de tablas largas.  En ese caso, en lugar del entorno \texttt{table} y el entorno \texttt{tabular} se usaría solamente el entorno \texttt{longtable}, que es una especie de híbrido de los dos, con un montón de características opcionales.  Para ilustrar su uso reproducimos \href{https://texblog.org/2011/05/15/multi-page-tables-using-longtable/}{un ejemplo de TeXblog} en la tabla~\ref{tab:tabla-larga}.

\begin{center}
\begin{longtable}{|c|c|c|c|}
\caption{Un ejemplo de tabla larga}
\label{tab:tabla-larga}\\
\hline
\textbf{Primera} & \textbf{Segunda} & \textbf{Tercera} & \textbf{Cuarta} \\
\hline
\endfirsthead
\multicolumn{4}{c}%
{\scriptsize\textbf{\tablename\ \thetable}\ -- \textit{Continúa de la página anterior}} \\
\hline
\textbf{Primera} & \textbf{Segunda} & \textbf{Tercera} & \textbf{Cuarta} \\
\hline
\endhead
\hline \multicolumn{4}{r}{\textit{\scriptsize Continúa en la página siguiente}} \\
\endfoot
\hline
\endlastfoot
1 & 2 & 3 & 4 \\ 1 & 2 & 3 & 4 \\ 1 & 2 & 3 & 4 \\ 1 & 2 & 3 & 4 \\
1 & 2 & 3 & 4 \\ 1 & 2 & 3 & 4 \\ 1 & 2 & 3 & 4 \\ 1 & 2 & 3 & 4 \\
1 & 2 & 3 & 4 \\ 1 & 2 & 3 & 4 \\ 1 & 2 & 3 & 4 \\ 1 & 2 & 3 & 4 \\
1 & 2 & 3 & 4 \\ 1 & 2 & 3 & 4 \\ 1 & 2 & 3 & 4 \\ 1 & 2 & 3 & 4 \\
1 & 2 & 3 & 4 \\ 1 & 2 & 3 & 4 \\ 1 & 2 & 3 & 4 \\ 1 & 2 & 3 & 4 \\
1 & 2 & 3 & 4 \\ 1 & 2 & 3 & 4 \\ 1 & 2 & 3 & 4 \\ 1 & 2 & 3 & 4 \\
1 & 2 & 3 & 4 \\ 1 & 2 & 3 & 4 \\ 1 & 2 & 3 & 4 \\ 1 & 2 & 3 & 4 \\
1 & 2 & 3 & 4 \\ 1 & 2 & 3 & 4 \\ 1 & 2 & 3 & 4 \\ 1 & 2 & 3 & 4 \\
\end{longtable}
\end{center}
\section{Ecuaciones} 
\label{sec:ecuaciones}

Si hay algo donde \LaTeX{} es especialmente útil, es en las fórmulas matemáticas.  Prácticamente no hay otra opción cuando las fórmulas son relativamente complejas.  En esta sección se muestran algunos ejemplos de ecuaciones.

\begin{equation}
    \mathbf{v} = \left[
    \begin{array}{c}
        2 \\
        3 \\
        -4 
    \end{array}
    \right]
\end{equation}

En \LaTeX{} es trivial el uso de cualquier notación de vectores.  Tan solo hay que familiarizarse con las órdenes correspondientes.  Por ejemplo, en esta ecuación:

\begin{equation} 
\vec{F} = m \vec{a}
\label{eq:dinamica}
\end{equation}

donde $\vec{F}$ es la fuerza, $\vec{a}$ es la actitud y $m$ la masa.

Las ecuaciones pueden refernciarse igual que las figuras, las tablas o las secciones.  Por ejemplo, la ecuación~\ref{eq:dinamica2} \ldots

\begin{equation} 
\alpha_{inicial} = \beta^{final} + \gamma
\label{eq:dinamica2}
\end{equation}

\begin{equation}
G(s)=\frac{(s^2+s+1)^2}{s^3+1}
\label{eq:dinamica3}
\end{equation}

El uso de letras griegas o símbolos matemáticos es también muy sencillo.  Tan solo hay que familiarizarse con la orden que los inserta.  Puede parecer difícil, pero basta con ojear una chuleta como \href{https://www.colorado.edu/physics/phys4610/phys4610_sp15/PHYS4610_sp15/Home_files/LaTeXSymbols.pdf}{ésta}\footnote{\url{https://www.colorado.edu/physics/phys4610/phys4610_sp15/PHYS4610_sp15/Home_files/LaTeXSymbols.pdf}}.

La ecuación~\ref{eq:transformacion} muestra un ejemplo de integral.  También es muy sencillo, puesto que la notación de los límites coincide con la de los subíndices y superíndices.

\begin{equation}
F(y) =  \int_{x_a}^{x_b} K(x,y) f(x) dx
\label{eq:transformacion}
\end{equation}

Cuando se necesita un entorno tabular dentro de un entorno matemático se utiliza el entorno \texttt{array}. La ecuación~\ref{eq:matriz} muestra un ejemplo.

\begin{equation}
\left(
\begin{array}{cccc}
1 & 0 & \cdots & 0 \\
0 & 1 & \cdots & 0 \\
\vdots & \vdots & \ddots & \vdots \\
0 & 0 & \cdots & 1
\end{array}
\right)
\label{eq:matriz}
\end{equation}

\section{Bibliografía, citas y referencias} 
\label{sec:bibliografia-citas}

Otro de los aspectos especialmente cuidados de \LaTeX{} es el manejo de bibliografía y citas.  En esta plantilla utilizamos el paquete \emph{natbib}.  Es un paquete muy flexible que permite adaptarse a casi cualquier estilo de citas existente.  Sin embargo, en los documentos de ingeniería suele haber bastante consenso en el estilo que hemos configurado en la plantilla.  A menos que tengas un motivo, no lo cambies.

La bibliografía en \LaTeX{} se hace con ayuda de unos archivos auxiliares escritos en formato BibTeX.  Es otro formato textual, con una serie de campos que hay que rellenar.  Para la composición de entradas BibTeX lo más sencillo es utilizar un editor online, como \url{http://truben.no/latex/bibtex/}.  Ten presente que algunos tipos de entradas pueden no estar configurados en el estilo de bibliografía que usas.  Por ejemplo, \emph{Online} y \emph{URL}, que aparecen en el editor en línea, no están en el estilo de bibliografía que usamos en esta plantilla.  Usa en su lugar \emph{Misc} como en el archivo \texttt{bib/how.bib} de esta plantilla.  En principio todas las entradas de bibliografía que utilices en tu TFG deben ponerse en \texttt{bib/main.bib}.

Con \LaTeX{} estándar se cita empleando la orden \texttt{cite} con el campo clave que contiene todo registro de BibTeX.  Esto puede valer, pero en el paquete \texttt{natbib}, se recomienda emplear \texttt{citep} en su lugar.  Por ejemplo, según el trabajo~\citep{armas2011estimation} \ldots mientras que según~\citep{castillo2010design} el control es una cosa muy buena. 

Con \texttt{natbib} tienes otra opción de cita, con la orden \texttt{citet}, que también se usa, especialmente cuando se quiere destacar el autor.  Por ejemplo, según \citet{castillo2011time} la potencia sin control no sirve de nada.  Puedes usar cualquiera de los dos estilos de cita, pero debes ser consistente.  La inconsistencia confunde al lector.

Una referencia bibliográfica se utiliza como argumento de autoridad, para dar peso a tu propia argumentación.  Por tanto, hay tres elementos clave que siempre deben estar: 
\begin{itemize}
    \item El autor, puesto que palabras anónimas no dan peso a nada.  Recuerda que el autor es lo que da peso a tu argumento.  No cites artículos divulgativos, ni autores sin un mínimo prestigio en el campo de lo que afirman.
    \item El título, puesto que el lector debe poder buscar por sí mismo el documento original.
    \item La fecha, puesto que un mismo autor puede cambiar de opinión a lo largo de su vida.  Por ejemplo  John Maynard Keynes es Premio Nobel pero tiene numerosos escritos contradictorios.  Su opinión era bastante cambiante con el tiempo.
\end{itemize}

Si falta alguno de estos elementos no es una referencia y no se cita.  Se puede poner como una nota a pié de página (\texttt{footnote}) o como una URL en el cuerpo del texto, pero no como una referencia.

Por cierto, es conveniente citar las fuentes.  Es decir, debes tomarte la molestia de buscar quién dijo o inventó lo que citas y dónde lo publicó por primera vez.  Es la mínima cortesía que se debe tener con los colegas de profesión.  Supongo que tú también querrás crédito por tu trabajo en tu futuro profesional.
		
\section{Hojas de datos}
\label{sec:hojas-datos}

Esto es un ejemplo de anexo. En la figura~\ref{fig:hoja-datos}\footnote{Fuente: \url{http://www.uclm.es}} se muestra la hoja de especificaciones del motor empleado.

\begin{figure}[ht!]
\centering
\fbox{\includegraphics[width=.95\textwidth]{RE30.pdf}}
\caption{Figura ejemplo}
\label{fig:hoja-datos}
\end{figure}	
\section{Código fuente} 
\label{sec:codigo-fuente}

Esto es un ejemplo de anexo. En este anexo se muestran algunos ejemplos de como plasmar código fuente.

En mi opinión, la forma más flexible y cómoda es usar el paquete \texttt{lstlistings}.  Permite incluir archivos o parte de archivos directamente del proyecto con la orden \texttt{lstinputlisting}.

\lstinputlisting[language=Matlab,
    caption={Ejercicio 20 como texto incorporado},
    label=src:ej20-input
]{memoria/tex/latex/ejercicio20.m}

O bien se puede copiar el texto del programa o fragmento en un entorno \texttt{lstlisting} con las mismas opciones que la orden \texttt{lstinputlisting}.

\begin{lstlisting}[language=Matlab,
    caption={Ejercicio 20 como texto en línea.},
    label=src:ej20-online
]
function [T]=Ejercicio20(f,c)

T = char('B'*ones(8,8));

for i=1:8
    for j=1:8
        if ( (i==f) || (j==c) || (i+j==f+c) || (i-j==f-c) )
            T(i,j)='*';
        elseif ( rem(i+j,2)~=0 )
            T(i,j)='N';
        end
    end
end

T(f,c)='R';
\end{lstlisting}

\info{Te recomendamos que incluyas los archivos o parte de los archivos directamente del código de tu proyecto, ya sea mediante \texttt{lstinputlisting} o mediante \texttt{inputminted}.  De esta forma mantendrás sincronizado el documento con el código fuente.}

\noindent El paquete \texttt{lstlisting} te permite quitar los números y el marco, cuando el código se incluye como parte del texto. 

\begin{lstlisting}[language=Matlab,
    frame=none,numbers=none
]
function [T]=Ejercicio20(f,c)

T = char('B'*ones(8,8));

for i=1:8
    for j=1:8
        if ( (i==f) || (j==c) || (i+j==f+c) || (i-j==f-c) )
            T(i,j)='*';
        elseif ( rem(i+j,2)~=0 )
            T(i,j)='N';
        end
    end
end

T(f,c)='R';
\end{lstlisting}

\noindent Otra forma de incluir código es mediante el entorno \texttt{verbatim}.  Este método no tiene resalte de sintaxis ni facilidades de ningún tipo para definir etiquetas o numerar las líneas. 

\begin{verbatim}
function [T]=Ejercicio20(f,c)

T = char('B'*ones(8,8));

for i=1:8
  for j=1:8
    if ( (i==f) || (j==c) || (i+j==f+c) || (i-j==f-c) )
      T(i,j)='*';
    elseif ( rem(i+j,2)~=0 )
      T(i,j)='N';
    end
  end
end

T(f,c)='R';
\end{verbatim}

\noindent Otra forma alternativa a \texttt{lstlisting} es el paquete \texttt{minted}, que colorea el programa según el lenguaje empleado.  Puede resultar más agradable desde el punto de vista estético pero recuerda que las fotocopias en color son entre 5 y 10 veces más caras que las fotocopias en blanco y negro.

\begin{minted}{matlab}
function [T]=Ejercicio20(f,c)

T = char('B'*ones(8,8));

for i=1:8
    for j=1:8
        if ( (i==f) || (j==c) || (i+j==f+c) || (i-j==f-c) )
            T(i,j)='*';
        elseif ( rem(i+j,2)~=0 )
            T(i,j)='N';
        end
    end
end

T(f,c)='R';
\end{minted}

