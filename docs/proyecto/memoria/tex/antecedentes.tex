\chapter{Motivación y antecedentes}
\label{ch:antecedentes}

El problema que pretendes resolver está dentro de un contexto que el cliente debe conocer.  Esta sección aporta información para conocer en detalle la importancia del problema y la dificultad para resolverlo con los productos y programas disponibles actualmente.

Este capítulo concentrará el grueso de las citas del TFG.  Dado que se trata del primer trabajo profesional, el alumno no suele estar familiarizado con las citas bibliográficas.  Pon toda tu atención en qué citas y cómo lo citas.  Revisa la sección~\ref{sec:bibliografia-citas} para las reglas mínimas que deben cumplir las citas.

\warning{Es muy importante respetar la regla de atribuir correctamente.  No es aceptable desde el punto de vista legal, ni tampoco desde el punto de vista ético, copiar trabajo de otros sin atribuirlo correctamente a los autores.}

Esta sección debe estudiar de forma sistemática todas las opciones ya disponibles en la actualidad para resolver el problema. No basta con una mera enumeración, hay que estudiarlos mínimamente para explicar por qué no son una solución para el problema o qué podría aportar a la solución del problema.

Un método sistemático para realizar esta parte del TFG es la revisión sistemática de literatura, conocida habitualmente por sus siglas en inglés SLR (\emph{Sistematic Literature Review}).  Un resumen muy sencillo de cómo realizar una SLR puede encontrarse en~\cite{anderskofodpetersen2014}.  También encontrarás consejos prácticos en~\cite{sandroschulze2017}.  Para un proceso más detallado, especialmente si tu problema tiene mucho arte previo, puedes consultar~\cite{barbarakitchenhamstuartcharters2007}.  Si no hay mucho arte previo pon este capítulo y el siguiente juntos.

En una tesis doctoral el análisis sistemático del estado del arte es esencial. En un TFG es importante, pero no hay que perder la cabeza.  Un TFG son unas 300 horas de trabajo de un estudiante medio que ya posea los conocimientos generales necesarios (volveremos a esto más tarde).  Considero que un buen análisis del estado del arte corresponde a un trabajo de entre 25 horas y 100 horas, dependiendo del tema del proyecto.  Si el tema es muy específico es más fácil hacer el estudio del estado del arte.

Termina este capítulo con una sección que resuma el estado del arte e identifique las lagunas lo más claramente posible.  Una tabla comparativa o un gráfico pueden ser formas interesantes de presentar la información.